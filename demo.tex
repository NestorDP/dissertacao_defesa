\documentclass[10pt]{beamer}

\usetheme[progressbar=frametitle]{metropolis}
\usepackage{appendixnumberbeamer}
\usepackage{booktabs}
\usepackage[scale=2]{ccicons}
\usepackage{pgfplots}
\usepgfplotslibrary{dateplot}
\usepackage{xspace}
\newcommand{\themename}{\textbf{\textsc{metropolis}}\xspace}

\usepackage{ragged2e}


\title{Comunicação entre Robot Operating System - ROS e SoC com FPGA integrado}
\subtitle{Universidade Federal da Bahia\\
            Programa de Pós-Graduação em Engenharia Elétrica}
\date{\today}
\date{Orientador: Prof. Dr. Wagner Luiz Alves de Oliveira\\Coorientador: Prof. Dr. Paulo César Farias}
\author{Autor: Nestor Dias Pereira Neto}
\institute{Salvador, 5 de dezembro de 2022}
\titlegraphic{\hfill\includegraphics[height=1.85cm]{imagens/logo.png}}


\begin{document}

\maketitle

\begin{frame}{Agenda}
  \setbeamertemplate{section in toc}[sections numbered]
  \tableofcontents[hideallsubsections]
\end{frame}


%=================================================================
% INTRODUÇÃO
%=================================================================
\section{Introdução}

% Slide CONTEXTUALIZAÇÃO I
%-----------------------------------------------------------------
\metroset{titleformat frame=smallcaps}
\begin{frame}{Contextualização}
	\begin{alertblock}{}
    	\begin{itemize}
		\setlength\itemsep{0.7em}
    	\item Projetos em robótica têm exigido maior percepção do ambiente.
    	\item Maior poder de processamento demanda maior consumo de energia.
    	\item FPGAs possuem potencial para melhorar desempenho de sistemas computacionais.
    	\item MACs de alta velocidade, processamento paralelo e baixas frequências de trabalho são comuns em FPGA. 
    	\item O uso do FPGA pode contribuir com ganho de poder de processamento associado ao baixo consumo.
    	\end{itemize}
    \end{alertblock}
 \nocite{LwIP,freertosbook,ROSeffect,PDSfpga,NiosIIbook,ROSfpga}
\end{frame}

% Slide CONTEXTUALIZAÇÃO II
%-----------------------------------------------------------------
\metroset{titleformat frame=smallcaps}
\begin{frame}{Contextualização}
	\begin{alertblock}{}
    	\begin{itemize}%[label=$\bullet$, itemsep=0.3cm]
		\setlength\itemsep{1em}
		\item Apesar de oferecer grande vantagens o FPGA em projetos de robótica é pouco incentivado.
    	\item Atualmente o framework ROS está se consolidando como o padrão na criação de novas plataformas robóticas.
    	\item O ROS é considerado um sistema operacional para robôs.
    	\item Desenvolver uma solução para estabelecer comunicação entre FPGA e o ROS.
    	\end{itemize}
    \end{alertblock}
 \nocite{LwIP,freertosbook,ROSeffect,PDSfpga,NiosIIbook,ROSfpga}
\end{frame}

% Slide PROBLEMA
%-----------------------------------------------------------------
\metroset{titleformat frame=smallcaps}
\begin{frame}[fragile]{Problema}
	\begin{center}
		\textbf{Como estabelecer a comunicação entre o ROS e um sistema de processamento auxiliar embarcado em um FPGA? }
	\end{center}
	\vspace{0.75cm}
	\begin{justify}
		Este problema é o que este trabalho busca resolver, possibilitando assim, o uso de aceleração por hardware através do FPGA no desenvolvimento de novos projetos de robótica.
	\end{justify}
\end{frame}

% Slide JUSTIFICATIVA
%-----------------------------------------------------------------
% \begin{frame}[fragile]{Justificativa}
%     \begin{itemize}
%         \item Nos últimos anos, novas técnicas para construção de robôs têm sido bastante estudadas e a robótica móvel tem recebido grande atenção.
%         \item Busca por maior autonomia, sistemais mais complexos.
%         \item FPGA uma opção para aumento de desempenho computacional, combinado com baixo consumo.
%         \item Poucas pesquisas relacionadas ao tema.
%         \item Reaproveitamento da pesquisa em outros projetos das mais diversas áreas.
%     \end{itemize}
% \end{frame}

% Slide OBJETIVO GERAL
%-----------------------------------------------------------------
\metroset{titleformat frame=smallcaps}
\begin{frame}{Objetivos}
    \begin{alertblock}{Objetivo geral}
		\vspace{0.25cm}
    	\begin{itemize}
    	\item Desenvolver uma solução para estabelecer comunicação entre \textit{Field-Programmable Gate Array - FPGA}, e o \textit{Robot Operating System - ROS}.
    	\end{itemize}
    \end{alertblock}
\end{frame}

% Slide OBJETIVOS ESPECÍFICOS
%-----------------------------------------------------------------
\metroset{titleformat frame=smallcaps}
\begin{frame}{Objetivos}
	\begin{alertblock}{Objetivos específicos}
        \begin{itemize}
        	\item Estudar teoria dos assuntos relevantes ao projeto: Verilog HDL, Embedded Linx,
			Cyclone V, TCP/IP Stack, ROS;
        	\item Estudar conceitos de programação de sockets em liguagem C++;
        	\item Implementar distribuição Embedded Linux para processador ARM embarcado no
			SoC Cyclone V da Intel;
        	\item Implementar distribuição Embedded Linux para processador ARM.
        	\item Estabelecer comunicação entre o ROS e o Cyclone V, através da tecnologia Gigabit
			Ethernet;
        	\item Avaliar o desempenho da rede entre o computador e o protótipo após a inclusão do
			FPGA ao sistema.
        \end{itemize}
	\end{alertblock}
\end{frame}


%=================================================================
% Parte I: Referenciais Teórico
%=================================================================
\section{Parte I: Referenciais Teórico}

% Slide SYSTEM-ON-CHIP
%-----------------------------------------------------------------
\metroset{titleformat frame=smallcaps}
\begin{frame}{Cyclone V SoC-FPGA}
	\begin{alertblock}{System-on-Chip}
		\vspace{0.1cm}
		\begin{justify}
			SoC é um acrônimo de \textit{System-on-a-Chip} ou apenas \textit{System on Chip}. Um SoC pode combinar diferentes elementos, em diferentes configurações, para formar um sistema completo.
		\end{justify}
		
		\begin{figure}[h]
			%\caption{{\footnotesize SoC genérico}}
			\begin{center}
				\includegraphics[scale=0.38]{imagens/basicsoc.png}\\
				{\footnotesize \textbf{Fonte:}}
			\end{center}
			\label{fig:SoC}
		\end{figure}
	\end{alertblock}
\end{frame}

% Slide FAMÍLIA CYCLONE V
%-----------------------------------------------------------------
\metroset{titleformat frame=smallcaps}
\begin{frame}{Cyclone V SoC-FPGA}
	\begin{alertblock}{Família Cyclone V SoC-FPGA Intel}
		\vspace{0.1cm}
		\begin{justify}
			A Intel fornece uma linha de produtos classificados como SoC-FPGA, os quais se aracterizam por possuir um rede de FPGA integrados a um processador ARM Cortex A9.
		\end{justify}
		\begin{figure}[h]
			%\caption{{\footnotesize SoC genérico}}
			\begin{center}
				\includegraphics[scale=0.29]{imagens/socfpga.png}\\
				{\footnotesize \textbf{Fonte:}}
			\end{center}
			\label{fig:SoC}
		\end{figure}
	\end{alertblock}
\end{frame}

% Slide INTERFACES HPS-FPGA
%-----------------------------------------------------------------
\metroset{titleformat frame=smallcaps}
\begin{frame}{Cyclone V SoC-FPGA}
	\begin{alertblock}{Interfaces HPS-FPGA}
		\vspace{0.1cm}
		\begin{justify}
			Esta estratégia de interconexão entre o HPS e o FPGA do Cyclone V em um único
			circuito integrado oferece.
		\end{justify}
		\begin{itemize}
        	\item Largura de banda de pico de mais de 100 Gbps;
        	\item Coerência de dados integrada;
        	\item Significativa economia de energia do sistema, eliminando caminhos de E/S externos
			entre o processador e o FPGA.
        \end{itemize}
	\end{alertblock}
\end{frame}



\metroset{titleformat frame=smallcaps}
\begin{frame}{Robot Operating System - ROS}
	\begin{alertblock}{Família Cyclone V SoC-FPGA}
		\vspace{0.1cm}
		\begin{justify}
			A Intel fornece uma linha de produtos classificados como SoC-FPGA, os quais se aracterizam por possuir um rede de FPGA integrados a um processador ARM.
		\end{justify}
		
		\begin{figure}[h]
			%\caption{{\footnotesize SoC genérico}}
			\begin{center}
				\includegraphics[scale=0.29]{imagens/socfpga.png}\\
				{\footnotesize \textbf{Fonte:}}
			\end{center}
			\label{fig:SoC}
		\end{figure}
	\end{alertblock}
\end{frame}
%=================================================================
% PARTE II: DESENVOLVIMENTO
%=================================================================
\section{Parte II: Desenvolvimento}

\begin{frame}{Metodologia}
    \begin{alertblock}{Procedimentos Metodológicos}
        A pesquisa será realizada em duas fases:
        %\metroset{block=fill}
        \begin{block}{Primeira fase:}
            \begin{itemize}
                \item Levantamento de informações teóricas sobre as tecnologias relacionadas com o tema.
            \end{itemize}
        \end{block}
    
        \begin{block}{Segunda fase}
            \begin{itemize}
                \item Serão desenvolvidos procedimentos, técnicas, algoritmos, circuitos e de todos os procedimentos práticos necessários para alcançar o objetivo da pesquisa.
            \end{itemize}
        \end{block}
    \end{alertblock}

  
\end{frame}

{
\metroset{titleformat frame=smallcaps}
\begin{frame}{Metodologia}
	\begin{alertblock}{Plano de trabalho}
	    \begin{itemize}
	        \item Será apresentado com detalhes nas metas físicas na seção cronograma.
	    \end{itemize}
	\end{alertblock}
	
	\begin{alertblock}{Materiais e infra-estrutura disponível}
	    \begin{itemize}
	        \item Para desenvolvimento do trabalho será utilizado o kit de desenvolvimento DE2-115
        da Terasic, que conta com um FPGA Intel EP4CE115 da família Cyclone IV. Inicialmente
        os teste com o ROS serão no ambiente de simulação Gazebo.
        \end{itemize}
	\end{alertblock}
	
\end{frame}
}

{
\metroset{titleformat frame=smallcaps}
\begin{frame}{Metodologia}
	\begin{alertblock}{Matérias cursadas}
	Todos os créditos obrigatórios com disciplinas já foram concluídos.
	    \begin{itemize}
	        \item Processamento Digitais de Sinais (PPGESP IFBA).
	        \item Processadores Digitais de Sinais - 8,5.
	        \item Inteligência Artificial - 8,0.
	        \item Robótica Móvel - 9,5.
	        \item Processamento Estatístico de Sinais - 8,4.
	        \item Componentes de Processadores Digitais de Sinais - 8,1.
	    \end{itemize}
	\end{alertblock}
	
\end{frame}
}

{
\metroset{titleformat frame=smallcaps}
\begin{frame}{Metodologia}
	\begin{alertblock}{Atividades desenvolvidas}
	Algumas atividades já foram concluídas.
	    \begin{itemize}
	        \item Revisão bibliográfica, estudo de trabalhos relacionado.
	        \item Conhecimento das ferramentas utilizadas.
	        \item Testes com sistema Nios II.
	        \item Implementação do FreeRTOS no NiosII.
	    \end{itemize}
	\end{alertblock}
	
\end{frame}
}


%=================================================================
% RESULTADOS
%=================================================================

\section{Parte III: Resultados}

{
\metroset{titleformat frame=smallcaps}
\begin{frame}{Cronograma}
	\begin{alertblock}{Metas físicas}
        \begin{enumerate}[1.]
        	\item Levantamento bibliográfico sobre os assuntos mais relevantes do projeto: ROS, Nios II, Verilog HDL, RTOS, TCP/IP Stack, Sockets.
        	\item Estudo detalhado do protocolo de comunicação entre os nós no ROS.
        	\item Desenvolvimento do sistema base do Nios II no Platform Designer.
        	\item Implementação do RTOS no sistema base.
        	\item Testes de comunicação TCP/IP entre o PC e o sistema embarcado no FPGA.
        	\item Desenvolvimento de uma aplicação de processamento de vídeo em FPGA.
        	\item Avaliação do desempenho do sistema proposto.
        	\item Elaboração da dissertação e publicação dos resultados.
        \end{enumerate}
	\end{alertblock}
\end{frame}
}

{
\metroset{titleformat frame=smallcaps}
\begin{frame}{Cronograma}

    \begin{table}[h]
    	\centering

    	\vspace{0.2cm}
    	\begin{tabular}{c|cccccccccccc}
    		\toprule
    		 & \multicolumn{12}{c}{{\tiny Meses}}\\ 
    		{\tiny Metas} & {\tiny 1} & {\tiny 2} & {\tiny 3} & {\tiny 4} & {\tiny 5} & {\tiny 6} & {\tiny 7} & {\tiny 8} & {\tiny 9} & {\tiny 10} & {\tiny 11} & {\tiny 12} \\ 
    		\midrule  
    		\midrule                           
    		{\tiny Levantamento Bibliográfico}& $\circledast$ & & & & & & & & & & & \\
    		\hline
    		{\tiny Estudo protocolos ROS} & & $\circledast$ & $\circledast$ & $\circledast$ & $\circledast$ & $\circledast$ & & & & & &  \\
    		\hline
    		{\tiny Desenv. do Nios II} & & & $\circledast$ & $\circledast$ & & & & & & & &  \\
    		\hline
    		{\tiny Implementação } {\tiny do RTOS} & & & & $\circledast$ & $\circledast$ & $\circledast$ & & & & & &  \\
    		\hline
    		{\tiny Testes de Comunicação} & & & & & & $\circledast$ & $\circledast$ & $\circledast$ & & & &  \\
    		\hline
    		{\tiny Desenv. do coprocessador} & & & & & & & & $\circledast$ & $\circledast$ & $\circledast$ & $\circledast$ &  \\
    		\hline
    		{\tiny Avaliação do desempenho} & & & & & & & & & & $\circledast$ & $\circledast$ & $\circledast$ \\
    		\hline
    		{\tiny Elaboração da } {\tiny dissertação} & & & & & & $\circledast$ & $\circledast$ & $\circledast$ & $\circledast$ & $\circledast$ & $\circledast$ & $\circledast$\\
    		\bottomrule 
    
    	\end{tabular}
    	\\
    	\label{tab:crono}
    \end{table}
\end{frame}
}

%{\setbeamercolor{palette primary}{fg=black, bg=yellow}
%\begin{frame}[standout]
%  Questions?
%\end{frame}
%}


\begin{frame}[allowframebreaks]{Referencias}

  \bibliography{demo}
  %\bibliographystyle{abbrv}
  \bibliographystyle{acm}
  
\end{frame}

\end{document}



%\begin{frame}{Blocks}
%  Three different block environments are pre-defined and may be styled with an
%  optional background color.

%  \begin{columns}[T,onlytextwidth]
%    \column{0.5\textwidth}
%      \begin{block}{Default}
%        Block content.
%      \end{block}

%      \begin{alertblock}{Alert}
%        Block content.
%      \end{alertblock}

%      \begin{exampleblock}{Example}
%        Block content.
%      \end{exampleblock}

%    \column{0.5\textwidth}

%      \metroset{block=fill}

%      \begin{block}{Default}
%        Block content.
%      \end{block}

%      \begin{alertblock}{Alert}
%        Block content.
%      \end{alertblock}

%      \begin{exampleblock}{Example}
%        Block content.
%      \end{exampleblock}
%
%  \end{columns}
%\end{frame}




%{%
%\setbeamertemplate{frame footer}{My custom footer}
%\begin{frame}[fragile]{Frame footer}
%    \themename defines a custom beamer template to add a text to the footer. It can be set via
%    \begin{verbatim}\setbeamertemplate{frame footer}{My custom footer}\end{verbatim}
%\end{frame}
%}